\section{Charakterystyka ciał niebieskich}

\begin{frame}
\subsection{Słońce}
\frametitle{Słońce}
Jedyna gwiazda w naszym układzie stanowiąca 99,9 całej jego masy, składa się ona głownie z helu i wodoru a jego temperatura sięga około 5505 °C . Słońce uformowało się w naszym układzie najszybciej spośród innych ciał a dokładniej 4,56 miliarda lat temu. Więc jest dopiero w około ¼ swojego życia, za następne 5 miliardów lat Słońce zacznie gwałtownie rosnąć stawiając się czerwonym olbrzymem jego promień wtedy gwałtownie urośnie według prognoz może nawet wchłonąć Ziemie aż straci całe swoje paliwo wtedy też rozpychanie się gwiazdy zacznie przegrywać z grawitacją i Słońce zacznie się gwałtownie kurczyć do wielkości ziemi tym samym zostając białym karłem. “Dzięki badaniom z dziedziny heliosejsmologi byliśmy w stanie podzielić naszą gwiazdę na następujące warstwy”
\end{frame}

\begin{frame}
\frametitle{Jądro}
\subsubsection{Jądro}
Według badaczy Jądro Słońca rozciąga się na około 20 jej całego promienia, składa się z plazmy rozgrzanej do około 15 milionów °C. “Analizy wskazują ze jądro obraca się szybciej niż zewnętrzna część strefy promienistej” [1] Wytwarza ono energie za pomocą syntezy jądrowej 
\end{frame}

\begin{frame}
\frametitle{Strefa promienista}
\subsubsection{Strefa promienista}
Obszar tuż nad jądrem, w którym synteza jądrowa nie zachodzi a transport energii następuje wyłącznie poprzez promieniowanie cieplne, “gęstość tego obszaru jest tak wielka, że światłu przebicie się przez tą warstwę zajmuje od 200 000 – 1000 000 lat”
\end{frame}

\begin{frame}
\frametitle{Strefa konwektywna}
\subsubsection{Strefa konwektywna}
Strefa konwektywna Strefa naszej gwiazdy, gdzie atomy są mniej zjonizowanie a sama materia jest mniej gęsta dzięki czemu ciepło unosi się w wyższe warstwy słońca, gdzie następuje “konwekcja termiczna” materia przechodzi do fotosfery, w której gęstość znowu wzrasta i materia znowu opada do strefy konwektywnej zapętlając cały proces. Wynik tego procesu można zaobserwować na powierzchni słońca, gdyż tworzy on na jego powierzchni “ślad w postaci granulacji i super granulacji”
\end{frame}

\begin{frame}
\frametitle{Fotosfera}
\subsubsection{Fotosfera}
Widzialna warstwa słońca z której wydostają się fotony docierające do Ziemi które my odbieramy jako światło i ciepło które nam daje
\end{frame}

\begin{frame}
\frametitle{Atmosfera}
\subsubsection{Atmosfera}
Słońce także posiada swoją atmosferę, którą możemy obserwować z Ziemi jako koronę słoneczną jest ona utrzymywana przez siły magnetyczne oraz grawitację. Nie należy bezpośrednio obserwować powierzchni słońca, gdyż grozi to utrata lub stałym uszczerbku wzroku służą do tego specjalne filtry nakładane na obiektywy teleskopów
\end{frame}

\begin{frame}
\frametitle{Merkury}
\subsection{Merkury}
Pierwsza i zarazem najmniejsza planeta skalista naszego układu z racji bliskości do Słońca powierzchnia planety osiąga wysoką choć nie największą temperaturę wynosząc średnio 167 °C. Z racji na jego bliskość do Słońca Merkury jest planetą trudną do obserwacji jednak da się ją zobaczyć gołym okiem można go jednak dojrzeć jedynie o wschodzie lub zachodzie słońca. “Dzień na planecie trwa 1407 godzin, czyli około 44 dni a rok 88 ziemskich dni” Planeta nie posiada żadnego naturalnego satelity. Brak atmosfery planety umożliwia dokładniejszą obserwacje jego powierzchni i znajdujących się na niej struktur terenu takich jak góry, uskoki czy kratery uderzeniowe jego powierzchnia jest bardzo zbliżona do naszego księżyca. Planeta mimo małego rozmiaru posiada silne pole magnetyczne. Planeta ma około 4,503 miliarda lat.
\end{frame}

\begin{frame}
\frametitle{Wenus}
\subsection{Wenus}
Druga planeta naszego układu rozmiarem zbliżonym do ziemi jest to najgorętsza planeta naszego układu o temperaturze przekraczającej 400°C jest to temperatura wystarczająca do stopienia ołowiu wiec badanie jej powierzchni za pomocą łazików stanowi duże wyzwanie technologiczne. Tak dużą temperaturę na Wenus zawdzięczamy jej atmosferze, w której znajduje się duża ilość gazów cieplarnianych. “Planeta nie posiada pola magnetycznego a jednak jej atmosfera nie uszczupla się co sugeruje, że aktywność wulkaniczna planety stale uzupełnia braki w gazach atmosfery” “Dzień na Wenus trwa 243 ziemskie dni a jej obieg wokół Słońca to 224,7 dni” więc planeta szybciej okrąża swoją gwiazdę niż się obraca wokół własnej osi. Większość jej powierzchni stanowią formacje wulkaniczne. Wiek planety to około 4,503 miliarda lat 
\end{frame}

\begin{frame}
\frametitle{Ziemia}
\subsection{Ziemia}
Planeta niezwykle unikalne w kosmicznej skali, jedyne miejsce, na którym potwierdzono życie, posiadająca głównie azotowo tlenową atmosferę utrzymująca naszą wodę w stanie ciekłym. Pierwsza planeta naszego układu posiadająca naturalnego satelitę. Ziemia posiada również własne pole magnetyczne które chroni nas przed wiatrem słonecznym. ¾ powierzchni naszej planety pokrywa woda posiadamy aktywne wnętrze naszej planety wprawiające w ruch płyty tektoniczne. Klimat na Ziemi na przestrzeni jej historii jest dość zmienny regularnie nasza planeta znacznie się ochładza pokrywając całą powierzchnię lodem. Na nasze szczęście przytrafiło się nam żyć w okresie najbardziej sprzyjającemu rozwojowi naszej cywilizacji
\end{frame}


\begin{frame}
\frametitle{Mars}
\subsection{Mars}
Ostatnia skalista planeta zwana czerwoną od swojego koloru powierzchni, Mars nie posiada atmosfery i pola magnetycznego. Na powierzchni Marsa spustoszenie sieją potężne burze piaskowe potrafiące objąć całą planetę. Temperatura marsa wacha się od -150 °C do 30°C zależnie jego odległości do Słońca. Mars może poszczycić się największym wzniesieniem w naszym układzie Olympus Mons bo o nim mowa to wulkan mający ponad 20km wysokości, osiągniecie tak dużej wielkości było możliwe dzięki braku tektoniki płyt na planecie przez co wulkan podczas erupcji wylewał magmę stale w jedno miejsce powodując nawarstwianie się wulkanu, nie przypomina on przez to typowej góry, które znamy z Ziemi, bo wchodząc na niego nie mielibyśmy wrażenia wchodzenia na wysoki obiekt tylko bardzo wysoki pagórek. Mars posiada także na swoich biegunach zamarznięte czapy lodu pod postacią gazów cieplarnianych. Planeta posiada także 2 księżyce Deimosa i Fobosa.
\end{frame}


\begin{frame}
\frametitle{Jowisz}
\subsection{Jowisz}
Największa planeta naszego układu gazowy gigant składający się głownie z wodoru i helu, posiada on bardzo dużo księżyców, bo aż 92 4 największe z nich nazywa się księżycami Galileuszowymi. Swoją dużą grawitacją działa dla nas niemal jak tarcza ściągając na siebie duże planetoidy, które mogłyby potencjalnie zagrozić naszej planecie. Powierzchnią Jowisza targają potężne burze jedna z najbardziej znanych to wielka czerwona plama jest ona tak wielka, że zmieściłaby się w niej cała nasza planeta, szacuje się, że trwa ona już od ponad 150 lat. Jowisz jak na swoją wielkość posiada bardzo dużą prędkość rotacyjną sprawiająca ze dzień trwa tam 9 godzin.
\end{frame}


\begin{frame}
\frametitle{Saturn}
\subsection{Saturn}
Drugi pod względem wielkości gazowy gigant posiadający niesamowity system 9 pierścieni składających się z kryształków lodu i skał. Posiada on równie imponującą liczbę księżyców, bo aż 82 z czego najbardziej znanym jest Tytan księżyc posiadający swoją własną atmosferę. Mimo swoich rozmiarów planeta jest mniej gęsta od wody co sprawia, że Saturn jest niczym wielka gazowa piłka. Na saturnie wieją bardzo szybkie wiatry, bo aż do 1800 km/h. System saturna poznaliśmy w największym stopniu dzięki wysłanym tam sondom takimi jak voyager które przez wiele lat dostarczały nam bezcennych danych o jego składzie i otoczeniu. Dzień na planecie trwa 11 godzin.
\end{frame}


\begin{frame}
\frametitle{Uran}
\subsection{Uran}
Pierwsza planeta odkryta za pomocą teleskopu przez Wiliama Herschela, wyróżnia się na tle innych planet swoją osią nachylenia, bo tam, gdzie inne planety mają równik Uran ma bieguny co sprawia ze nie istnieje tam pojęcie dnia i nocy. Jedynie co 40 lat, kiedy planeta okrąża Słońce oświetlana jest druga jej połowa. Powierzchnia Urana składa się głownie, z płynnego lodu które obtacza jego skaliste jądro. Swój kolor planeta zawdzięcza dużej zawartości metanu w jego atmosferze. Nie wiele ludzi wie, ale Uran posiada swój własny system pierścienie niestety są one bardzo ciemne i niewidoczne z Ziemi przez co nie możemy podziwiać ich piękna. “Uran posiada aż 27 księżyców a obrót planety wokół własnej osi trwa 17 godzin. Jego temperatura utrzymuje się w granicach -197°C”
\end{frame}


\begin{frame}
\frametitle{Neptun}
\subsection{Neptun}
Planeta równie niebieska co Uran odkryta jednak dzięki wyliczeniom matematycznym astronomów którym nie zgadzały się obserwacje Urana z zapiskami, uznano wtedy, że na wpływ orbity Urana działa inna planeta, czyli nasz bohater. Neptun podobnie jak Jowisz posiada wielką ciemną plamę  która jednak jest powodowana przez jego system chmur co sprawia ze plama pojawia się i znika w różnych częściach planety. Ciekawostką jest ze na Neptunie mogą padać deszcze diamentów  powodowanych przez ogromne ciśnienie, na planecie które uwalnia węgiel z metanu. Neptun także posiada system pierścieni dużo informacji na ten temat dała nam sonda kosmiczna voyager 2 jest to jedyny jak na razie obiekt, który wysłaliśmy na tą planetę. Jednym z jego 14 księżycy jest Tryton. Planeta posiada pole magnetyczne a dzień na tej planecie trwa 16 godzin.
\end{frame}


\begin{frame}
\frametitle{Pluton}
\subsection{Pluton}
Pluton “Do 2006 roku planeta naszego układu następnie zdegradowana do miana planety karłowatej” Pluton utracił swój status przez to ma zbyt małą masę i tym samym nie pozbył się ze swojego otoczenia z innych obiektów. Obiekt ten jest mniejszy od naszego księżyca da rade go zaobserwować z pomocą odpowiednio dużego teleskopu. Pluton wraz ze swoją bliźniaczą planetą karłowatą Charonem orbituje wokół ich wspólnego środka masy. Obrót Plutona wokół własnej osi zajmuje 153 godziny. Pluton posiada również swoje księżyce a jest ich aż 4.
\end{frame}

\begin{frame}
\frametitle{Pas Kuipera}
\subsection{Pas Kuipera}
Pas Kuipera Obszar w układzie słonecznym w którym znajduje się wiele małych obiektów: komet, planetoid i planety karłowate jak np. Pluton i Charon. Skład znajdujących się tam obiektów zawiera głownie lód przez małą ilość promieni słonecznych jakie tam dopierają ciała te mogą mieć tylko stałą postać. 
\end{frame}


